% Time-stamp: <09/10/02 01:57:13 vilhuber>
% $Id: Presentation-PSD.tex 3219 2012-09-27 07:47:11Z vilhu001 $

% normal line:
\documentclass[xcolor=table,compress]{beamer}
% to create notes:
%\documentclass[handout,notes=only]{beamer}
% to create handouts
%\documentclass[xcolor=table,handout,compress]{beamer}
% to create a different kind of handouts
%\documentclass{article}
%\usepackage[envcountsect]{beamerarticle}

%\setbeameroption{handout}
%\setbeameroption{show notes}


%
% Packages
%
\mode<article> % only for the article version
{
  \usepackage{fullpage}
  \usepackage{hyperref}
}
\usepackage{ifpdf}
\ifpdf
\usepackage{embedfile}
\embedfile{\jobname.tex}
\fi

\usepackage{graphicx}
%\usepackage{pstricks}
\usepackage{xcolor}
\usepackage{pifont}
%\usepackage{../chicago}
\usepackage{pgf}
\usepackage{amsmath,amssymb,amsfonts}
\usepackage[latin1]{inputenc}
\usepackage{colortbl}
\usepackage[english]{babel}
\usepackage{array}
\usepackage{pdfpages}
% usage:
%   \includepdf[pages={1}]{myfile.pdf}
%   \includepdf[pages={1,3,5}]{myfile.pdf} would include pages 1, 3, and 5 of the file. 
%   To include the entire file, you specify pages={-}, where {-}
%\usepackage{landscape}
\usepackage{listings}
\lstloadlanguages{R,bash,SAS}
\lstset{numbers=left, stepnumber=1, basicstyle=\tiny}

%\usepackage{lmodern}
%\usepackage[T1]{fontenc}

\usepackage{times}
%\usepackage{colortbl}

%============================================================
% Beamer specific styles and configs
%============================================================

\mode<presentation>
{
% alternative, could always use
%\usetheme{Census}
\usetheme{cornell}
\useoutertheme{cornell}
}


%\setbeamercovered{dynamic}



%============================================================
% Title
%============================================================

\title[Computing for Economists]{Workshop: High-performance computing for economists}
\author[Vilhuber, Abowd, Mansfield, McKinney]{%
  Lars~Vilhuber\inst{1} \and
  John M. Abowd\inst{1} \and
  Richard~Mansfield\inst{1} \and
  Hautahi~Kingi\inst{1} \and
  Flavio~Stanchi\inst{1} \and
  Sida~Peng\inst{1} \and
  Kevin~L.~McKinney %\inst{2}%
}

\institute[Cornell]{
  \inst{1}%
   Cornell University, Economics Department,
%\and \inst{2} U.S. Census Bureau
}%
\date[August 18-21, 2014]{August 18-21, 2014: Day 2}
\subject{HPC}


% % % % % % % % % % % % % % % % % Main document
\begin{document}
\frame{\titlepage}
\section{Intro}
\section{Basics}
\section[VCS]{Version control systems}
\section{Subroutines}

\subsection{Goals}
\begin{frame}{Basic subroutine programming}
\begin{block}{Goal}
\begin{itemize}
\item Show the basics of proper subroutine programming
\item Advantages, pitfalls
\item Examples in R
\item Later: generalization and differences in other programming languages
\end{itemize}
\end{block}
\end{frame}
\subsection[Control]{Control structures}
\begin{frame}{Control structures in programming languages}
\small
\begin{block}{Mostly generic}
\begin{itemize}
\item \texttt{if, else}: testing a condition [R, SAS] 
\item \texttt{for}: execute a loop a fixed number of times [R, in SAS: \texttt{do}]
\item \texttt{while}: execute a loop while a condition is true [R,SAS]
\item \texttt{until}: execute a loop until a condition is true [SAS]
\item \texttt{repeat}: execute an infinite loop [R]
\item \texttt{break}: break the execution of a loop [R, SAS]
\item \texttt{next}: skip an interation of a loop [R]
\item \texttt{return}: exit a function [R]
\end{itemize}
\end{block}
\end{frame}





\subsubsection{if}
\begin{frame}[fragile]{Control structures: if}
\begin{columns}
\begin{column}[t]{.48\textwidth}
\pause
\begin{block}{... in R}
\lstset{language=R}
\begin{lstlisting}
if(<condition>) {
## do something
} else {
## do something else
}
if(<condition1>) {
## do something
} else if(<condition2>) {
## do something different
} else {
## do something different
}
\end{lstlisting}
\end{block}
\end{column}
\hfill\pause
\begin{column}[t]{.48\textwidth}
\begin{block}{... in SAS}
\lstset{language=SAS}
\begin{lstlisting}
if (<condition>) then do;
## do something
end; else do;
## do something else
end;
if (<condition1>) then do;
## do something
else if (<condition2>) then do;
## do something different
end; else do;
## do something different
end;
\end{lstlisting}
\end{block}
\end{column}
\end{columns}
\end{frame}




\subsubsection{for}

\begin{frame}[fragile]{Control structures: for}
Run through a fixed sequence of numbers (or in R, a sequence of vectors)
\begin{columns}
\begin{column}[t]{.48\textwidth}
\pause
\begin{block}{simple loop in R}
\lstset{language=R}
\begin{lstlisting}
for(i in 1:10) {
print(i)
}
\end{lstlisting}
\end{block}
\end{column}
\hfill\pause
\begin{column}[t]{.48\textwidth}
\begin{block}{... in SAS}
\lstset{language=SAS}
\begin{lstlisting}
do i = 1 to 10;
put i;
end;
\end{lstlisting}
\end{block}
\end{column}

\end{columns}
\end{frame}


\begin{frame}[fragile]{Control structures: for}
Across programming languages, some flexibility:
\begin{columns}
\begin{column}[t]{.48\textwidth}
\pause
\begin{block}{Equivalent loops in R}
\lstset{language=R}
\begin{lstlisting}
x <- c("a", "b", "c", "d")
for(i in 1:4) {
print(x[i])
}
for(i in x) {
print(i)
}
for(i in 1:4) print(x[i])
\end{lstlisting}
\end{block}
\end{column}
\hfill
\begin{column}[t]{.48\textwidth}
\begin{block}{... in SAS}
\lstset{language=SAS}
\begin{lstlisting}
do i = 1 to 10;
put i;
end;
\end{lstlisting}
\end{block}
\end{column}

\end{columns}
\end{frame}

\subsubsection{Breaking out}
\begin{frame}{Interrupting loops}
\begin{block}{How a loop ends}
\begin{itemize}
\item at the end
\item by resetting the counter to the end value, or by setting the looping condition to its exit value explicitly
\item when it encounters a \texttt{break} (for \texttt{repeat} loops in R)
\end{itemize}
\end{block}
\end{frame}



\begin{frame}[fragile]{What do we loop over?}
\begin{block}{Within the loop, something is done}
\lstset{language=R, basicstyle=\footnotesize}
\begin{lstlisting}
for (i in 1:100000) {
	## do something here 
	# stuff (1 line)
	# stuff (2nd line)
	# done
	}
\end{lstlisting}
\end{block}
\pause
\begin{block}{What if that is really complicated?}
\lstset{language=R, basicstyle=\footnotesize}
\begin{lstlisting}
for (i in 1:100000) {
	## do something really complicated here (398 lines)
	}
\end{lstlisting}
\end{block}
\end{frame}

\subsection{Subroutines}
\begin{frame}[fragile]{Subroutines}
\begin{block}{Breaking into discrete chunks}
\begin{columns}
\begin{column}{.48\textwidth}
\lstset{language=R, basicstyle=\tiny}
\begin{lstlisting}
for (i in 1:100000) {
	## do something really complicated here
	## (398 lines)
	}
\end{lstlisting}

\end{column}\hfill
\pause
\begin{column}{.48\textwidth}
In the previous example, this was unwieldy (which loop are you closing on line 400?)
\end{column}
\end{columns}
\pause
\begin{columns}[b]
\begin{column}{.48\textwidth}
We can start by breaking out the complicated stuff into a well-defined subroutine.
\end{column}
\pause
\hfill
\begin{column}{.48\textwidth}
\lstset{language=R, basicstyle=\tiny}
\begin{lstlisting}
for (i in 1:100000) {
	## new subroutine
	do_something_complicated(args=something)
	##
	}
\end{lstlisting}
\end{column}
\end{columns}
\end{block}
\end{frame}


\subsubsection{Definition}
\begin{frame}{Subroutines, procedures, etc.}
\begin{block}{What do you call a subroutine?}
``The same things in different [programming] languages can have different names. Programs, Procedures, Functions, \textbf{Subroutines}, Subprograms, Subqueries ... these words all have very similar meanings. [...]
We can call an object, it executes and performs a complex process. Whether that object is called any one of the above list depends on what programming language it is written in, whether a human being can call it, or whether it has to be called by another program or one of the other names on the list.''
\end{block}
{\tiny \href{http://en.wikibooks.org/wiki/Computer_Programming/Procedures_and_functions}{Source}}
\end{frame}

\begin{frame}{An practical overview of subroutine programming}
\begin{block}{We will use R...}
... but expand to cover what this might look like in SAS (macros), Stata (programs, ado files), Matlab (functions) tomorrow.

... use this as a building block to scale to HPC.
\end{block}
\end{frame}



\begin{frame}
\begin{block}{You already use them!}
Much functionality in SAS, Stata, Matlab, R is implemented as a subroutine:
\begin{itemize}
\item \texttt{proc import} in SAS (compare call to log file)
\item Most statistical functions in Stata (\texttt{regress} used earlier is actually in \texttt{../ado/base/r/regress.ado})
\end{itemize}
\end{block}
You can create them too!
\end{frame}


\begin{frame}[fragile]{Functions in R}
\begin{block}{R objects of class "function"}
\lstset{language=R}
\begin{lstlisting}
f <- function(<arguments>) {
	## do something here
	}
\end{lstlisting}
\begin{itemize}
\item Functions can be arguments to other functions
\item Functions can be nested (even recursive)
\item Functions can have named arguments with default values, or missing values
\end{itemize}
\end{block}
\vfill
\tiny We draw on Peng's "Computing for Data Analysis, Week 2" for this section.
\end{frame}

\lstset{language=R}
\subsubsection[Examples]{Basic examples}
\begin{frame}[fragile]{Basic examples}
\begin{block}{Let's start with a simple example}
%\lstset{caption=\lstname}
\lstinputlisting{../programs/day1/3-1-simple-function.R}
\end{block}
\pause
The function is defined, has two arguments with default values, and a really short name.
\pause
\begin{block}{Results}
\begin{lstlisting}
> source("3-1-simple-function.R")
> f
function(a=1,b=0) {
       rnorm(a,mean = b)
}
> f()
[1] 0.1709822
> set.seed(10)
> f(a=10,b=5)
 [1] 5.018746 4.815747 3.628669 4.400832 5.294545 5.389794 3.791924 4.636324
 [9] 3.373327 4.743522
\end{lstlisting}
\end{block}
\end{frame}


\subsubsection[Creatin]{Creating functions}
\begin{frame}{Defining function arguments}
Functions have \textit{named} arguments which potentially have \textit{default} values.
\begin{itemize}
\item The formal arguments are the arguments included in the function definition
\item Not every function call in R makes use of all the formal arguments
\item Function arguments can be missing, or can have default values
\end{itemize}
\end{frame}


\begin{frame}[fragile]{Function arguments at invocation}
Function arguments can be passed by position (\textit{positional}) or by explicit name (in which case, position doesn't matter).
\begin{lstlisting}[caption=Equivalent calls,basestyle=\normalsize]
> set.seed(10)
> mean(f(b=5,a=10))
[1] 4.509343
> set.seed(10)
> mean(f(a=10,b=5))
[1] 4.509343
> set.seed(10)
> mean(f(10,5))
[1] 4.509343
\end{lstlisting}
However, it is \textbf{good practice} to use \textit{named} arguments, as they make use of the function (especially when more complex) easier and more transparent.
\end{frame}



\begin{frame}[fragile]{Lazy evaluation}
\begin{block}{Arguments can remain unused:}

\begin{lstlisting}[numbers=none]
f <- function(a, b) {
a^2
}
> f(2)
[1] 4
\end{lstlisting}
\texttt{b} was never used.
\end{block}
\begin{block}{Arguments can be faulty, but don't lead to problems until used }
\begin{lstlisting}[numbers=none]
f <- function(a, b) {
print(a)
print(b)
}
> f(45)   # note no value provided for b
[1] 45
Error in print(b) : argument "b" is missing, with no default
>
\end{lstlisting}
The absence of a value for \texttt{b} only led to an error at the time it was called.
\end{block}
\end{frame}

\begin{frame}[fragile]{Scoping}
\begin{block}{Scope of functions}

\begin{lstlisting}[numbers=none]
f <- function(a,b) {
   print(a)
   print(c)
   c <- paste(a,b)
   print (c)
}
\end{lstlisting}
\lstset{numbers=left}
Try it out:\pause
\begin{lstlisting}
> f(c('a'),c('b'))
[1] "a"
function (..., recursive = FALSE)  .Primitive("c") <== ERROR!
[1] "a b"
\end{lstlisting}
\end{block}
\end{frame}

\begin{frame}[fragile]{Scoping}
\begin{block}{What happened?}
\texttt{c} was not defined, leading to the error on line 4. Line 5 reports what happens to \texttt{c} once it is defined.
\end{block}
\begin{block}{Understanding the scope}
\begin{lstlisting}[firstnumber=last]
> c <- c('Nothing')
> f(c('a'),c('b'))
[1] "a"
[1] "Nothing"
[1] "a b"
\end{lstlisting}
So what value does \texttt{c} now have?\pause
\begin{lstlisting}[firstnumber=last]
> print(c)
[1] "Nothing"
\end{lstlisting}
\end{block}
\end{frame}


\begin{frame}{Scoping}
\begin{block}{Understanding the scope of variables}
\begin{itemize}[<+->]
\item In the example, \texttt{c} was defined both inside the function and outside. 
\item R took the ``global'' value of \texttt{c} ...
\item until the function defined it, at which point it took a new value internal to the function
\item ... which was only used until the function ended. The ``global'' value had not changed.
\end{itemize}
\end{block}
\begin{block}{Other programming languages}
Each programming language has a different way of handling these. Take care to read up on it.
\end{block}
\end{frame}




\begin{frame}{Searching for variables (and functions)}
\begin{itemize}
\item The global environment or the user?s workspace is always the first element of the search list and the \textbf{base} package is always the last.
\item The order of the packages on the search list matters!
\item User?s can configure the order of packages as they get loaded on startup ...
\item When a user loads a package with library the namespace of that package gets put in position 2 of the search list (by default) and everything else gets shifted down the list.
\end{itemize}
\end{frame}

\begin{frame}{Scoping (name resolution) rules}
\begin{block}{Lexical scoping}
\textbf{Lexical} (or \textbf{static}) resolution can be determined at compile time, and is also known as \textbf{early binding}
\end{block}
\begin{block}{Dynamic scoping}
\textbf{Dynamic} resolution can in general only be determined at run time, and thus is known as \textbf{late binding}. 
\end{block}
Most modern languages use lexical scoping for variables and functions, though de facto dynamic scoping is common in macro languages, which do not directly do name resolution. \href{http://en.wikipedia.org/wiki/Scope_(computer_science)}{[1]}
\end{frame}


\begin{frame}[fragile]{Scoping rules}
\begin{block}{Back to R}
\begin{lstlisting}[numbers=none]
f <- function(a,b) {
   print(paste(a,b))
   print(c)
}
\end{lstlisting}
\begin{itemize}
\item Named (formal) arguments (\texttt{a,b}) are always local 
\item Free variables (not defined in the call) (\texttt{c}) are subject to scoping rules
\end{itemize}
\end{block}
\end{frame}


\begin{frame}{Lexical scoping in R}
Lexical scoping in R means that 
\begin{quote}
the values of free variables are searched for in the environment in which the
function was defined.
\end{quote}
What is an environment?
\begin{quote}
\begin{itemize}
\item An environment is a collection of (symbol, value) pairs, i.e. x is a symbol and
3.14 might be its value.
\item Every environment has a parent environment; it is possible for an environment to
have multiple ?children?
\item A function + an environment = a closure or function closure.
\end{itemize}
\end{quote}
\end{frame}





\begin{frame}{Safe use of scoping}
\begin{block}{Define all variables as arguments}
To avoid confusion about where variables are defined, unless you have a good reason to deviate from this
\begin{itemize}
\item Define all variables as arguments to the function
\item In other languages, explicitly define the scope of a variable
\end{itemize}
\end{block}
\end{frame}




\begin{frame}[fragile]{Scoping in SAS}
\begin{block}{Scope of macro variables in SAS}
Define a similar program in SAS
\begin{lstlisting}[language=SAS,numbers=none]
%macro myprogram(a=,b=);
%let c=&a.&b.;
%put &c.;
%mend;
\end{lstlisting}
\pause Call it and assess where \texttt{c} comes from:
\begin{lstlisting}[language=SAS,numbers=none]
%myprogram(a=a,b=b);
%put &c.;
%let c=Nothing;%put &c.;
%myprogram(a=a,b=b);
%put &c.;
\end{lstlisting}
\end{block}
\end{frame}


\begin{frame}[fragile]{Scoping in SAS}
with results
\begin{lstlisting}[language=SAS,numbers=none]
1    %macro myprogram(a=,b=);
2    %let c=&a.&b.;
3    %put &c.;
4    %mend;
6    %myprogram(a=a,b=b);
ab
WARNING: Apparent symbolic reference C not resolved.
7    %put &c.;
&c.
8    %let c=Nothing;%put &c.;
Nothing
9    %myprogram(a=a,b=b);
ab
10   %put &c.;
ab
\end{lstlisting}

\end{frame}



\begin{frame}[fragile]{Scoping in SAS}
Better:
\begin{lstlisting}[language=SAS,numbers=none]
1    %macro myprogram(a=,b=);
2    %local c;
3    %let c=&a.&b.;
4    %put &c.;
5    %mend;
6    %let c=Nothing;
7    %myprogram(a=a,b=b);
ab
8    %put &c.;
Nothing
\end{lstlisting}
\end{frame}


\begin{frame}{Efficient use of scoping}
\begin{block}{Differences that need to be taken into account}
\begin{itemize}
\item Scoping rules can be leveraged to improve optimization (see Peng's Coursera course and others)
\item The use of scoping differs across languages (what is feasible in R cannot be simply translated into Java or SAS or Stata)
\end{itemize}
\end{block}
\end{frame}


\begin{frame}[fragile]{Naming rules}
\begin{block}{Naming is important}
\begin{itemize}
\item Naming is important - both of functions and of arguments: Compare the following two functions:
\begin{lstlisting}
f <- function(a,b,c) {
   x <- sample(y,c)
   lm(a ~ b , data=x )
}
\end{lstlisting}
and
\begin{lstlisting}
sample_reg <- function(lhs,rhs,samplesize=10,data=) {
   subset <- sample(data,samplesize)
   lm(lhs ~ rhs , data=subset )
}
\end{lstlisting}
\item Give functions and variables meaningful names
\end{itemize}
\end{block}
\end{frame}


\begin{frame}[fragile]{Naming rules}
\begin{block}{Robustness is important}
\begin{itemize}[<+->]
\item Think about backward compatibility (and your program library). Say you first used this:
\begin{lstlisting}
sample_reg <- function(lhs,rhs,samplesize=10,data=) {
   subset <- sample(data,samplesize)
   lm(lhs ~ rhs , data=subset )
}
sample_reg(earnings,education,5,data=cps)
\end{lstlisting}
\item Now you extend your model
\color{red}
\begin{lstlisting}
sample_reg <- function(lhs,rhs,samplesize=10,data=) {
   subset <- sample(data,samplesize)
   lm(log(lhs) ~ rhs , data=subset )
}
sample_reg(earnings,education,5,data=cps)
\end{lstlisting}
\end{itemize}
\end{block}
\end{frame}


\begin{frame}[fragile]{Naming rules}
\begin{block}{Robustness is important}
\begin{itemize}
\item You should either give your function a different name, or make the call robust to both the ``old'' way and the ``new'' way:\color{blue}
\begin{lstlisting}
sample_reg <- function(lhs,rhs,samplesize=10,data=,logs = false ) {
   subset <- sample(data,samplesize)
   if ( logs ) {
   	_lhs <- log(lhs)
   	}
   	else {
   	_lhs <- lhs
   	}
   lm(_lhs ~ rhs , data=subset )
}
\end{lstlisting}\color{black}
which will work for both calls... 
\end{itemize}
\end{block}
\end{frame}





\subsubsection[Power]{The power of functions}
\begin{frame}
\vfill
The power of functions
\vfill
\end{frame}













\begin{frame}[fragile]{The power of functions}
\begin{block}{Why bother with functions?}
\begin{itemize}
\item Initial example: putting 398 command lines into separate file (ease of use)
\item Expansion on that: re-using the function across multiple projects (function library)
\item Your function is a complete specification. You know want to vary or perturb all 25 parameters slightly, for robustness checks.
\begin{lstlisting}
for (i in 1:1000) {
  for (j in 1:1000) {
  	my_regression(model=base,xi=i,xj=j)
  }
}
\end{lstlisting}
\end{itemize}
\end{block}
\end{frame}




\begin{frame}[fragile]{The power of functions}
\begin{block}{Why bother with functions?}
\begin{itemize}
\item You want to database the results:
\begin{lstlisting}
for (i in 1:1000) {
  for (j in 1:1000) {
  	results_db[i,j]=my_regression(model=base,xi=i,xj=j)
  }
}
\end{lstlisting}
\end{itemize}
\end{block}
\end{frame}




\begin{frame}[fragile]{The power of functions}
\begin{block}{The ultimate power of functions: scaling}
\begin{itemize}
\item You want to speed the whole thing up by parallelizing:
\begin{lstlisting}
library(doMC)
registerDoMC()
foreach (i = 1:1000, .combine=cbind) %dopar% {
  for (j in 1:1000) {
  	results_db[i,j]=my_regression(model=base,xi=i,xj=j)
  }
}
\end{lstlisting}
which will run 1000 parallel threads, on as many cores as you can.
(see \href{http://cran.r-project.org/web/packages/doMC/vignettes/gettingstartedMC.pdf}{doMC vignette} or the help in your Rstudio installation)
\end{itemize}
\end{block}
\end{frame}


\subsubsection{Conclusion}

\begin{frame}{Take-away}
\begin{block}{Major points}
\begin{itemize}
\item Subroutines are a powerful tool to write clean, understandable code
\item Subroutines (functions, macros, programs) are present in some form in all statistical programming languages
\item Use consistent, clear naming
\item Use robust subroutines, expanding them into a library that you can use and share across projects
\item Clean subroutines are a critical component to scaling your analysis (parallelization)
\end{itemize}
\end{block}
\end{frame}


\begin{frame}{Take-away}
\begin{block}{Additional items}
\begin{itemize}
\item learn how to debug (different in each language, also critical to scaling)
\item subroutines don't magically make your code efficient - they allow you to figure out which portions are not
\end{itemize}
\end{block}
\end{frame}


\end{document}