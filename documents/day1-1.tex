% Time-stamp: <09/10/02 01:57:13 vilhuber>
% $Id: Presentation-PSD.tex 3219 2012-09-27 07:47:11Z vilhu001 $

% normal line:
\documentclass[xcolor=table,compress]{beamer}
% to create notes:
%\documentclass[handout,notes=only]{beamer}
% to create handouts
%\documentclass[xcolor=table,handout,compress]{beamer}
% to create a different kind of handouts
%\documentclass{article}
%\usepackage[envcountsect]{beamerarticle}

%\setbeameroption{handout}
%\setbeameroption{show notes}


%
% Packages
%
\mode<article> % only for the article version
{
  \usepackage{fullpage}
  \usepackage{hyperref}
}
\usepackage{ifpdf}
\ifpdf
\usepackage{embedfile}
\embedfile{\jobname.tex}
\fi

\usepackage{graphicx}
%\usepackage{pstricks}
\usepackage{xcolor}
\usepackage{pifont}
%\usepackage{../chicago}
\usepackage{pgf}
\usepackage{amsmath,amssymb,amsfonts}
\usepackage[latin1]{inputenc}
\usepackage{colortbl}
\usepackage[english]{babel}
\usepackage{array}
\usepackage{pdfpages}
% usage:
%   \includepdf[pages={1}]{myfile.pdf}
%   \includepdf[pages={1,3,5}]{myfile.pdf} would include pages 1, 3, and 5 of the file. 
%   To include the entire file, you specify pages={-}, where {-}
%\usepackage{landscape}

%\usepackage{lmodern}
%\usepackage[T1]{fontenc}

\usepackage{times}
%\usepackage{colortbl}

%============================================================
% Beamer specific styles and configs
%============================================================

\mode<presentation>
{
% alternative, could always use
%\usetheme{Census}
\usetheme{cornell}
\useoutertheme{cornell}
}


%\setbeamercovered{dynamic}



%============================================================
% Title
%============================================================

\title[Computing for Economists]{Workshop: High-performance computing for economists}
\author[Vilhuber, Abowd, Mansfield, McKinney]{%
  Lars~Vilhuber\inst{1} \and
  John M. Abowd\inst{1} \and
  Richard~Mansfield\inst{1} \and
  Kevin~L.~McKinney %\inst{2}%
}

\institute[Cornell]{
  \inst{1}%
   Cornell University, Economics Department,
%\and \inst{2} U.S. Census Bureau
}%
\date[August 20-22, 2013]{August 20-22, 2013: Day 1}
\subject{HPC}


% % % % % % % % % % % % % % % % % Main document
\begin{document}
\frame{\titlepage}

\section[Intro]{Introduction}
\begin{frame}
\begin{block}{What do you learn in a Ph.D. program?}
\pause
 How to learn...
\end{block}
\end{frame}

\begin{frame}
\begin{block}{Goal of this class}
\pause
To open new doors, to be able to conceive of problems that you didn't think had a feasible solution.\newline
\pause
To broaden your knowledge about what you do NOT know
\end{block}
\end{frame}

\subsection{Overview}
\begin{frame}{Overview}
\begin{block}{Day 1}
\begin{itemize}[<+->]
\item Programming basics (Lars)
\begin{itemize}
\item Choosing an editor
\item How to structure programs, texts, etc.
\item A clean sequence of programs
\item NX, SSH, Linux, request an account on cluster
\item Basic scripting
\end{itemize}
\item  Basics of version control (Lars)
\begin{itemize}
\item File-system based version control
\item More formal version control (Subversion, Git)
\item Working with servers
\item Setting up infrastructure at Cornell
\end{itemize}
\item Subroutines: the example of function programming in R (Lars)
\end{itemize}
\end{block}
\end{frame}

\begin{frame}{Overview}
\begin{block}{Day 2}

\end{block}
\end{frame}

\begin{frame}{Overview}
\begin{block}{Day 3}

\end{block}
\end{frame}

\subsection{Structure}

% % % % % % % % % % % % % % % End overview
\section[Basics]{Programming basics}
\subsection[Editors]{Choosing editors}
\begin{frame}{Choosing an editor}
\begin{block}{... or system}
\begin{columns}[t]
\begin{column}{.48\textwidth}
\color{red}
Separate editors and systems 
\begin{itemize}
\item MS Word and math editor
\item \href{http://www.libreoffice.org}{LibreOffice}
\item \LaTeX (TeXstudio, TeXMaker, Scientific Workplace, \href{http://www.tug.org/texworks/}{TeXWorks}+\href{http://miktex.org/}{Miktex}, etc.)
\item NotePad++ (Windows)
\item Gedit, (X)Emacs (Linux)
\end{itemize}
\end{column}
\hfill
\begin{column}{.48\textwidth}
\color{blue}
Integrating programming and running
\begin{itemize}
\item \href{http://en.wikipedia.org/wiki/Integrated_development_environment}{IDE} ( \href{http://en.wikipedia.org/wiki/Eclipse_(software)}{Eclipse}, \href{http://en.wikipedia.org/wiki/ActiveState_Komodo}{ActiveState Komodo}, etc.)
\item Native programming GUIs (SAS, Matlab, Stata)
\item Gedit, (X)Emacs 
\end{itemize}
Integrating programs and text/results
\begin{itemize}
\item SWeave (integrates \LaTeX \ and R)
\item RStudio (GUI to R and SWeave)
\item StatRep (Integrated SAS and \LaTeX, \href{http://support.sas.com/resources/papers/proceedings12/324-2012.pdf}{Source 1}, \href{http://support.sas.com/StatRepPackage}{Source 2})
\end{itemize}
\end{column}
\end{columns}


\end{block}

\end{frame}


\subsection{Basic scripting}
\begin{frame}[fragile]{Basic scripting in Linux}
\begin{block}{A basic loop on the command line}
\begin{verbatim}
content...
\end{verbatim}

\end{block}
\end{frame}


\begin{frame}
\tiny{Source: \href{http://www.cyberciti.biz/faq/bash-for-loop/}{[1]}}
\end{frame}
\begin{frame}
Now let's try it out
\end{frame}

\section[VCS]{Version control systems}
\begin{frame}
\href{day1-2.pdf}{Next section}
\end{frame}

\section{Subroutines}
\begin{frame}
\href{day1-3.pdf}{Next section}
\end{frame}

\end{document}